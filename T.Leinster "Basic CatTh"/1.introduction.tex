% !TeX root = ./document.tex
\documentclass[document]{subfiles}
\begin{document}
\chapter{Введение}
Основным концептом в этом конспекте является понятие \textit{универсального} свойства. Приведём несколько примеров.

\begin{example}
	Обозначим множество из одного (неважно какого) элемента за $1$. Тогда $1$ имеет следующее свойство:
	\[\forall X ~ \text{ существует единственное отображение из } X \text{ в } 1\]
\end{example}

\begin{proof}
	Пусть $X$ --- множество. Тогда существует отображение $f: X \to 1$, т.к. его мы можем определить так:
	\[\forall x : f(x) \text{ равно элементу множества } 1\]

	Такое отображение единственно, т.к. любое отображение $f: X \to 1$ для каждого $x \in X$ ставит в соответствие единственный элемент из множества $1$.
\end{proof}
В данном конспекте под словом \textit{кольцо} будем подразумевать кольцо с нейтральным элементом по умножению. 
Так же все \textit{гомоморфизмы} подразумевают не только сохранение операций сложения и умножения, но и сохранение нейтрального элемента (то есть если $f$ --- гомоморфизм, то для любых колец $R, Q$ выполняется: $f(1_R) = 1_Q$)

\begin{example}
    Пусть $R$ --- кольцо. Тогда существует единственный гомоморфизм $f: \mathbb{Z} \to R$
\end{example}

\begin{proof}

    \underline{Существование}
    Положим функцию:
    \[f(n) = \begin{cases} \underbrace{1 + \dots + 1}_n ~~~ n > 0 \\ 0 ~~~~~~~~~~~~~~~~ n = 0 \\ -f(-n) ~~~~~~~ n < 0\end{cases}\]

    Очевидно это является гомоморфизмом.

    \underline{Единственность}
    Пусть $f, g$ --- гомоморфизмы из $\mathbb{Z}$ в $R$. Тогда по свойству сохранения нейтрального элемента:
    \[g(n) = g(\underbrace{1 + \dots + 1}_n) = \underbrace{g(1) + \dots + g(1)}_n = \underbrace{1 + \dots + 1}_n = f(n)\]
    для всех $n > 0$

    Так же гомоморфизм сохраняет нейтральный элемент по сложению: $g(0) = f(0)$

    А так же имеем для $n < 0$:

    \[g(n) = -g(-n) = -f(-n) - f(n)\]

\end{proof}

\textit{По сути}, может существовать только один объект удовлетворяющий универсальному свойству. Здесь "\textit{по сути}" означает с точностью до изоморфизма. То есть если два объекта удовлетворяют одному универсальному свойству, то они обязательно изоморфны. Например:

\begin{lemma}
    Пусть $A$ --- кольцо со свойством: для всех колец $R$ существует единственный гомоморфизм $f: A \to R$. Тогда $A \cong \mathbb{Z}$
\end{lemma}
\begin{proof}
    Кольцо $A$ будем называть \textit{начальным}. В прошлом примере мы доказали, что $\mathbb{Z}$ является начальным кольцом.
    
    Из определения имеем, что существуют единственные гомоморфизмы $\varphi: A \to \mathbb{Z}$ и $\varphi': \mathbb{Z} \to A$
    Композиции из гомоморфизмов:
    \[\varphi' \circ \varphi:A \to A, ~~~~~~ \varphi \circ \varphi': \mathbb{Z} \to \mathbb{Z} \]
    сами являются гомоморфизмами, и по прошлому утверждению получаем, что они единственны, а значит:
    
    \[\varphi' \circ \varphi = \text{id}_A ~~~~~~ \varphi \circ \varphi' = \text{id}_\mathbb{Z}\]
    Поэтому гомоморфизмы $\varphi, \varphi'$ являются взаимообратными, то есть изоморфизмами.
\end{proof}

Перейдём к векторным и топологическим пространствам.

\begin{example}
    Для любого множества $S$ существуют векторное пространство $V$ и функция (являющаяся индексацией базиса): $i: S \to V$ обладающие универсальным свойством:
    \begin{center}
        Для любого векторного пространства $W$ и для любой функции $f: S \to W$ существует единственная линейная функция $\bar{f}: V \to W$, такая что $f = \bar{f} \circ i$
    \end{center}
    Перепишем это в виде диаграммы:
    $$
    \begin{tikzcd}
	    S \arrow[r, "i"] \arrow[dr, "{\forall \text{ функции }} f"'] & V \arrow[d, dashed, "{\exists! \text{ линейная }} \bar{f}"] \\
	    & \forall W
    \end{tikzcd}
    $$
\end{example}
\begin{proof}
    Это потому что любая функция определённая на базисных векторах однозначно расширяется до линейной функции.
\end{proof}

\begin{example}
    Пусть $S$ множество. Положим функцию $$i: S \to D(S) ~~~(i(s)=s)$$ где $D(S)$ --- топологическое пространство с дискретной топологией, построенное на $S$. Тогда функция $i$ и пространство $D(S)$ имеют универсальное свойство:
    \begin{center}
        Для любого топологического пространства $X$ и для любой функции $f: S \to X$ существует единственная непрерывная функция $\bar{f}: D(S) \to X$, такая что $f = \bar{f} \circ i$
    \end{center}
\end{example}
\begin{proof}
    Перепишем утверждение в виде диаграммы:
    $$
    \begin{tikzcd}
	    S \arrow[r, "i"] \arrow[dr, "{\forall \text{ функции }} f"'] & D(S) \arrow[d, dashed, "{\exists! \text{ непрерывная }} \bar{f}"] \\
	    & \forall X
    \end{tikzcd}
    $$

    Непрерывность любой функции $D(S) \to X$ является очевидной, в силу дискретной топологии. 

    \underline{Существование}
    Положим $\bar{f}(s) = f(s)$, тогда необходимо проверить условие, что $\bar{f} \circ i = f$:
    \[(\bar{f} \circ i) (s) = \bar{f}(i(s)) = \bar{f}(s) = f(s)\]

    \underline{Единственность}
    Из-за условия $\bar{f} \circ i = f$ имеем, что:
    \[\forall s \in S: (\bar{f} \circ i)(s) = \bar{f}(s) = f(s)\]
    Т.к. $f$ зафиксировано, $\bar{f}$ обязан быть единственным
\end{proof}

\begin{example}
    Пусть $U, V$ --- любые векторные пространства. Тогда существуют такое векторное пространство $T$ и соответствующее ему билинейное отображение $b: U \times V \to T$, обладающие универсальным свойством:
    \begin{center}
        Для любого векторного пространства $W$ и для любого билинейного отображения $f: U \times V \to W$ существует единственное линейное отображение $\bar{f}: T \to W$, такое что $f = \bar{f} \circ b $
    \end{center}
    Можно представить в виде диаграммы:
    $$
    \begin{tikzcd}
	    U \times V \arrow[r, "b"] \arrow[dr, "{\forall \text{ функции }} f"'] & T \arrow[d, dashed, "{\exists! \text{ непрерывная }} \bar{f}"] \\
	    & \forall W
    \end{tikzcd}
    $$
\end{example}

\begin{lemma}
    Пусть $U, V$ --- векторные пространства. Положим $b: U \times V \to T$, $b': U \times V \to T'$ --- билинейные отображения с универсальным свойством. Тогда $T \cong T'$.
\end{lemma}
\begin{proof}
    Покажем связи в виде диаграммы:
    $$
    \begin{tikzcd}
        & T \arrow[d, "j"] \\
        U \times V \arrow[ur, "b"] \arrow[r, "b'"] \arrow[dr, "b", swap] & T' \arrow[d, "j'"] \\
        & T
    \end{tikzcd}
    $$
    Тогда по универсальному свойству для $b, T$ подставляя $b', T'$ имеем единственное линейное отображение $j: T \to T'$

    Аналогично по универсальному свойству для $b', T'$ подставляя $b, T$ имеем единственное линейное отображение $j': T' \to T$

    Тогда имеем $j' \circ j: T \to T$ линейное отображение, для которого выполнено: $(j' \circ j) \circ b = b$, тогда по универсальному свойству для $b, T$ подставляя $j' \circ j, T$ имеем, что $j' \circ j = \text{id}_T$. Аналогично $j \circ j' = \text{id}_{T'}$. Получили что $j$ --- изоморфизм.
\end{proof}

Это доказательство своего рода "шаблон", который применяется для доказательства изоморфности объектов, обладающих общим универсальным свойством.

\end{document}
