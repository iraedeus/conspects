% !TeX root = ./document.tex
\documentclass[document]{subfiles}
\begin{document}
\chapter{Теория вероятностей}

\section{Основы теории множеств и аксиоматика вероятности}

Здесь основными концепциями являются: пространтство элементарных исходов, события, теоретико-множественные операции, сигма алгебра и аксиомы вероятности.

\begin{problem}
    Для каждого из следующих экспериментов, опишите пространство элементарных исходов:
    \begin{enumerate}
        \item Подбрасывание монетки 4 раза.
        \item Подсчёт количества листьев растения, пострадавших от насекомых.
        \item Подсчитать время жизни (в часах) конкретного бренда лампочек.
        \item Записать веса десятидневных крыс.
        \item Посчёт доли дефектов в партии электронных компонент.
    \end{enumerate}
\end{problem}
\begin{solution}
    \begin{enumerate}
        \item Обозначим событие "Выпал орел" как $O$, а событие "Выпала решка" как $P$, тогда пространство элементарных исходов $S$ состоит из 16 элементов: $S = \{OOOO, OOOP, \dots, OPPP, PPPP\}$
        \item Подмножество неотрицательных целых чисел.
        \item Подмножество вещественных положительных чисел.
        \item Подмножество вещественных положительных чисел.
        \item Отрезок $[0,1]$.
    \end{enumerate}
\end{solution}

\begin{problem}
    Докажите следующие равенства:
    \begin{enumerate}
        \item $A \setminus B = A \setminus (A \cap B) = A \cap B^c$
        \item $B = (B \cap A) \cup (B \cap A^c)$
        \item $B \setminus A = B \cap A^c$
        \item $A \cup B = A \cup (B \cap A^c)$
    \end{enumerate}
\end{problem}
\begin{solution}
    Доказывается по определению равенства множеств.
\end{solution}

\begin{problem}
    Докажите следующие свойства событий:
    \begin{enumerate}
        \item Коммутативность пересечения и объединения.
        \item Ассоциативность пересечения и объединения.
        \item Правила де Моргана.
    \end{enumerate}
\end{problem}
\begin{solution}
    Докажу только правила де Моргана.
    \begin{enumerate}
        \item $(A \cup B)^c = A^c \cap B^c$

            $\subset$

            Пусть $x \in (A \cup B)^c$ значит $x \not\in A \cup B$, значит $x\not\in A$ и $x \not in B$, а значит по итогу:
            \[x \in A^c \text{ и } x \in B^c \implies x \in A^c \cap B^c\]

            $\supset$

            Пусть $x \in A^c \cap B^c$ тогда $x \in A^c$ и $x \in B^c$ значит $x \not\in A,B$ то есть:
            \[x \not\in A \cup B \implies x \in (A \cup B)^c\]
    \end{enumerate}
\end{solution}

\begin{problem}
    Пусть $A$ и $B$ --- события. Найдите формулы в терминах $P(A), P(B), P(A \cap B)$ для следующих событий:
    \begin{enumerate}
        \item Либо $A$ либо $B$ либо $A \cap B$.
        \item Либо $A$ либо $B$, но не оба сразу.
        \item Хотя бы один из $A$ или $B$.
        \item Не более одного из $A$ или $B$.
    \end{enumerate}
\end{problem}
\begin{solution}
 
    \begin{enumerate}
        \item Назовем событие $C = A \cup B$, и $D = A \cap B$ тогда наше искомое событие это $C \cup D$, причем $D \subset C$, тогда:
            \[P(C \cup D) = P(C)\]
            Т.к. мы не знаем, пересекаются ли события, то просто разложить в сумму нельзя. Тогда положим:
            \begin{gather*}
                A = (A \setminus (A \cap B)) \cup (A \cap B) \\
                B = (B \setminus (A \cap B)) \cup (A \cap B)
            \end{gather*}
            тогда:

            \[P(A) + P(B) = P(A \setminus (A \cap B)) + P(B \setminus (A \cap B)) + 2 P(A \cap B)\]
            
            однако с другой стороны:

            \[A \cup B = (A \setminus (A \cap B)) \cup (B \setminus (A \cap B)) \cup (A \cap B)\]

            то есть:

            \[P(A \cup B) = P(A) + P(B) - P(A \cap B)\]
        
        \item Имеем событие $(A \cup B) \cap (A \cap B)^c$, преобразуем:
            \begin{gather*}
                 (A \cup B) \cap (A \cap B)^c = (A \cup B) \cap (A^c \cap B^c) = (B \setminus A) \cup (A \setminus B) = \\
                 = (A \setminus (A \cap B)) \cup (B \setminus (A \cap B)) 
            \end{gather*}

            Посчитаем вероятность:

            \[P((A \cup B) \cap (A \cap B)^c) = P(A \setminus (A \cap B)) + P(B \setminus (A \cap B)) + P(A \cap B) - P(A \cap B)\]

            Но по первому  пункту, мы знаем, что:
            \[A \cup B = (A \setminus (A \cap B)) \cup (B \setminus (A \cap B)) \cup (A \cap B)\]

            а значит:
            \[P((A \cup B) \cap (A \cap B)^c) = P(A \cup B) - P(A \cap B) = P(A) + P(B) - 2 P(A \cap B)\]
            
        \item То же самое что и в первом пункте: $P(A \cup B) = P(A) + P(B) - P(A \cap B)$

        \item Имеем событие $C = (A \cup B)^c \cup ((A \cup B) \cap (A \cap B)^c)$, тогда по 2 пункту и свойствам вероятности:
            \begin{gather*}
                P(C) = P((A \cup B)^c) + P(A) + P(B) - 2P(A \cap B) = \\
                = 1 - P(A \cup B) + P(A) + P(B) - 2P(A \cap B) = \\
                = 1 - (P(A) + P(B) - P(A \cap B)) + P(A) + P(B) - 2P(A \cap B) = \\ 
                = 1 - P(A \cap B)
            \end{gather*}

    \end{enumerate}
\end{solution}

\begin{problem}
    Две монеты подбрасываются независимо, для каждой монеты $P(\text{Решка}), P(\text{Орёл})$ равна $u$ и $w$ соответственно. Определим:
    \begin{gather*}
        p_0 = P(\text{Решка выпала 0 раз}) \\ 
        p_1 = P(\text{Решка выпала 1 раз}) \\ 
        p_2 = P(\text{Решка выпала 2 раз}) \\ 
    \end{gather*}
    Можно ли выбрать $u, w$ таким образом, чтобы $p_0 = p_1 = p_2$?
\end{problem}
\begin{solution}
    Понятно, что $p_0 = w^2, ~ p_1 = 2uw, ~ p_2 = u^2$
    $$
    \begin{cases}
        u^2 = 2uw \\ 
        2uw = w^2 
    \end{cases} \implies
    \begin{cases}
        u = 2w \\
        2u = w
    \end{cases} \implies
    4w = w
    $$
    что очевидно невозможно.
\end{solution}

\begin{problem}
    Сформулируйте и докажите правила де Моргана для конечной совокупности множеств $A_1, \dots, A_n$
\end{problem}
\begin{solution}
    Очевидно.
\end{solution}

\begin{problem}
    Пусть $\{A_\alpha : \alpha \in \Gamma\}$ --- счётное или несчётное множество. Докажите, что:
    \begin{enumerate}
        \item $(\bigcup_\alpha A_\alpha)^c = \cap_\alpha A^c_\alpha$
        \item $(\bigcap_\alpha A_\alpha)^c = \cup_\alpha A^c_\alpha$
    \end{enumerate}
\end{problem}
\begin{solution}
    Аналогично
\end{solution}

\begin{problem}
    Пусть $S$ --- пространство элементарных исходов.
    \begin{enumerate}
        \item Покажите, что множество $\mathcal{B} = \{\varnothing, S\}$ является сигма-алгеброй.
        \item Пусть $\mathcal{B} = \mathcal{P}(S)$. Докажите, что $\mathcal{B}$ --- сигма-алгебра
        \item Докажите что пересечение двух сигма-алгебр является сигма-алгеброй.
    \end{enumerate}
\end{problem}
\begin{solution}
    \begin{enumerate}
          \item Очевидно множество замкнуто относительно операции дополнения: $\varnothing^c = S, ~ S^c = \varnothing$ 
               
              Возьмем любое счётное объединение, очевидно оно равно либо $\varnothing$ либо $S$

          \item С дополнением все понятно. Замкнутость счётного объединения следует из замкнутости этой операции на множестве всех подмножеств.

          \item Пусть $\mathcal{B}_1, \mathcal{B}_2$ -- сигма-алгебры. 

              Возьмем его пересечение: $\mathcal{B} = \mathcal{B}_1 \cap \mathcal{B}_2$.
              Пусть $A \in \mathcal{B}$, тогда ${A \in \mathcal{B}_1, \mathcal{B}_2}$. Тогда из того, что $\mathcal{B}_1, \mathcal{B}_2$ --- сигма-алгебры, следует:
            \[A^c \in \mathcal{B}_1, \mathcal{B}_2 \implies A^c \in \mathcal{B}\]

            Аналогично и для объединения

    \end{enumerate}
\end{solution}

\begin{problem}
    Покажите, что:
    \begin{enumerate}
        \item Из аксиомы счётной аддитивности следует аксиома конечной аддитивности.
        \item Пусть $A_1 \supset \dots \supset A_n \supset \dots$ бесконечная последовательность стремящаяся к $\varnothing$. Положим аксиому непрерывности:
            \[A_n \underset{n \to \infty}{\longrightarrow} \varnothing \implies P(A_n) \to 0\]
            Докажите, что из аксиомы непрерывности и аксиомы конечной аддитивности следует аксиома счётной аддитивности.
    \end{enumerate}
\end{problem}
\begin{solution}
    \begin{enumerate}
        \item Пусть $A_1, \dots, A_n$ --- конечная последовательность попарно несовместных событий. Можем её дополнить ${A_i = \varnothing, i > n}$, тогда мы получаем бесконечную последовательность попарно несовместных события ${A_1, \dots, A_n, \varnothing, \dots}$. Тогда по аксиоме счётной аддитивности:
            \[P\left(\bigcup_{i=1}^n A_i\right) = P\left(\bigcup_{i=1}^\infty A_i\right) = \sum_{i=1}^\infty P(A_i) = \sum_{i=1}^n P(A_i)\]

        \item Положим $R_n = \bigcup_{i=n}^\infty A_i$, тогда:
            \[\bigcup_{i=1}^\infty A_i = \left( \bigcup_{i=1}^{n-1} A_i \right) \cup R_n, ~~~ \forall n\]

            Т.к. $R_n$ и $\bigcup_{i=1}^{n-1} A_i$ не пересекаются, то по конечной аддитивности имеем:
            \[P\left(\bigcup_{i=1}^\infty A_i\right) = P(R_n) + \sum_{i=1}^{n-1} P(A_i)\]

            Из того, что $R_n \to \varnothing$ можем перейти к пределу в равенстве, тогда по аксиоме непрерывности получаем:
            \[P\left(\bigcup_{i=1}^\infty\right) A_i = \sum_{i=1}^\infty P(A_i)\]
    \end{enumerate}
\end{solution}

\begin{problem}
    Пусть $P(A) = 1/3$ и $P(B^c) = 1/4$, могут ли быть несовместными события $A$ и $B$?
\end{problem}
\begin{solution}
    Пусть могут, тогда $A \cap B = \varnothing$. Из этого сразу следует, что $A \subset B^c$. Однако $P(A) > P(B^c)$, что противоречит теореме $1.2.9$ из книги.
\end{solution}

\begin{problem}
Пусть $P(\cdot)$ --- вероятность. Докажите, что если ${P(B) > 0}$, то ${P(\cdot ~|~ B)}$ является вероятностью (то есть удовлетворяет аксиомам Колмогорова).
\end{problem}
\begin{solution}
    
\end{solution}

\section{Комбинаторные методы и классическое определение вероятности}

Здесь основными концепциями являются: правило умножения, сочетания и размещения, выборки с возвращением и без возвращения

\begin{problem}
    Пусть пространство элементарных исходов $S$ имеет $n$ элементов. Докажите, что количество его подмножеств равно $2^n$
\end{problem}
\begin{solution}
Пусть $A$ --- некоторое подмножество $S$. Тогда для каждого элемента из $S$ можно ввести характеристическую функцию ${f: S \to \{0, 1\}}$:
    \[f(s) = \begin{cases}1, s \in A \\ 0, s \not \in A \end{cases}\]

    Таким образом, получаем $2^n$ различных функций.
\end{solution}

\begin{problem}
    Завершите доказательство теоремы $1.2.14$ для случая $k > 2$.
\end{problem}
\begin{solution}
    Пусть для теорема верна для $k$ задач, тогда если $n_i$ --- количество способом решить задачу $i$, то всего вариантов решения:
    \[n_1 \cdot \dots \cdot n_k\]

    Возьмем $k+1$-ю задачу, которую можно решить $n_{k+1}$ способами, тогда $k+1$ задач можно решить:
    \[\underbrace{(n_1 \cdot \dots \cdot n_k) + \dots + (n_1 \cdot \dots \cdot n_k)}_{n_{k+1}} = n_1 \cdot \dots \cdot n_k \cdot n_{k+1}\]

    По индукции теорема верна для любого $k$.
\end{solution}

\begin{problem}
    Какое количество инициалов может быть (английский алфавит), если каждый человек имеет только одну фамилию и:
    \begin{enumerate}
        \item Только два имени
        \item Одно или два имени
        \item Одно, два или три имени
    \end{enumerate}
\end{problem}
\begin{solution}
    \begin{enumerate}
        \item Первой буквой фамилии может быть любая буква из 26 букв алфавита. То же самое можно сказать и про два имени. Тогда общее количество равно $26^3$
        \item Пусть у человека одно имя. Тогда способов --- $26^2$. Пусть у человека два имени, тогда способов $26^3$. Эти множества не пересекаются, значит их способы нужно сложить: $26^3 + 26^2$
        \item Аналогично получаем $26^4 + 26^3 + 26^2$
    \end{enumerate}
\end{solution} 

\begin{problem}
    В игре "Домино", каждое домино может быть сформировано из 2 чисел. Сами домино являются симметричными (например (2, 6) = (6, 2).
    Какое количество домино может быть сформировано, используя числа $1, \dots, n$?
\end{problem}
\begin{solution}
    Разобъем случаи на два непересекающиеся: случай, когда числа разные и случай, когда одинаковы. Тогда первый случай возможен $n(n-1)/2$ способами, а второй $n$ способами. Сложим:
    \[n + \frac{n(n-1)}{2} = n\left(\frac{2}{2} + \frac{n-1}{2}\right) = n(n+1)/2\]
\end{solution}

\begin{problem}
    Пусть $n$ шаров расположено случайно на $n$ клетках. Какая вероятность того, что будет свободна только одна клетка?
\end{problem}
\begin{solution}
    Всего способов расположить шары --- $n^n$. Зафиксируем пустую ячейку, сделать это можно $n$ способами. По принципу Дирихле одна из оставшихся ячеек будет занята двумя шарами, эту ячейку можно выбрать $n-1$ способом, а росположить в нее шары $\binom{n}{2}$ способами. Наконец оставшиеся различные шары можно разложить по $n-2$ ячейкам $(n-2)!$ способами. Соберем все вместе:
    \[n \cdot (n-1) \cdot \frac{n!}{2! (n-2)!} \cdot (n-2)!\]

    Получаем ответ:
    \[\frac{n! \binom{n}{2}}{n^n}\]
\end{solution}

\begin{problem}
    Многомерная функция с непрерывными производными имеет свойство: порядок взятия производных не важен. Тогда:
    \begin{enumerate}
        \item Как много производных четвертого порядка имеет функция трёх переменных?
        \item Докажите, что функция $n$ переменных имеет $\binom{n+r-1}{r}$ производных порядка $r$
    \end{enumerate}
\end{problem}
\begin{solution}
    \begin{enumerate}
        \item Представим 1, 2, 3 и 4-ые производные как маркеры, каждый из которых можем подставить к каждой из переменной. По аналогии со стенками из раздела комбинаторики имеем 2 стенки и 4 маркера. Тогда вариантов:
        \[\frac{6!}{2!4!} = 15\]
        \item По аналогии с прошлым пунком имеем $r$ маркеров и $n - 1$ стенок, тогда способов:
            \[\frac{(n + r - 1)!}{(n-1)! r!} = \binom{n + r - 1}{r}\]
    \end{enumerate}
\end{solution}

\begin{problem}
    Телефон звонит 12 раз в неделю. Какова вероятность того, что каждый день будет происходить как минимум один звонок?
\end{problem}
\begin{solution}
    Используется формула включений-исключений. Вычислений много слишком.
\end{solution}

\begin{problem}
    Шкаф содержит $n$ пар обуви. Если $2r$ обуви выбираются случайным образом $(2r < n)$, какая вероятность того, что не найдется ни одной пары?
\end{problem}
\begin{solution}
    Всего вариантов выбрать $2r$ обуви --- $\binom{2n}{2r}$. Пусть не нашлось ни одной пары, значит каждый раз мы берём обувь из разных пар, которых $n$ штук. Таких вариантов --- $\binom{n}{2r}$. Нужно учесть то, что для каждой выбранной обуви можно было выбрать либо левую, либо правую, значит всего таких вариантов $2^{2r}$. Итоговый ответ:
    \[\frac{2^{2r} \cdot \binom{n}{2r}}{\binom{2n}{2r}}\]
\end{solution}

\begin{problem}
    \begin{enumerate}
        \item В лотерее, содержащей 366 дней года (включая 29 февраля), какова вероятность того, что первые 180 дней, выбранных (без возвращения), равномерно распределены по 12 месяцам?
        \item Какова вероятность того, что среди первых 30 выбранных дней не будет ни одного дня из сентября?
    \end{enumerate}
\end{problem}
\begin{solution}
    \begin{enumerate}
        \item Всего вариантов выбрать из 366 дней 180 --- $\binom{366}{180}$. Если мы хотим равномерного распределения дней по месяцам, то в каждом месяце должно быть $180/12 = 15$ дней. Всего 31 дневных месяцев --- 7, 30 дневных --- 4, а 29 дневных --- 1. Тогда итоговый ответ:
            \[\frac{\binom{31}{15}^7 \cdot \binom{30}{15}^4 \cdot \binom{29}{15}}{\binom{366}{180}}\]
        \item Всего вариантов --- $\binom{366}{30}$. Благоприятных исходов --- $\binom{336}{30}$ и ответ:
            \[\frac{\binom{366}{30}}{\binom{336}{30}}\]
    \end{enumerate}
\end{solution}

\begin{problem}
    Два человека подбрасывают честную монетку $n$ раз. Найдите вероятность того, что будет выброшено одинаковое количество орлов.
\end{problem}

\section{Условная вероятность, независимость и формула Байеса}

Здесь основными концепциями являются: условная вероятность, независимость событий, формула полной вероятности и формула Байеса.

\begin{problem}
    Примерно треть всех близнецов --- однояйцевые, соответственно две трети --- разнояйцевые. Однояйцевые близнецы обязательно одного пола, причем мужчины и женщины одинаково вероятны. Среди разнояйцевых близнецов примерно одна четвертая --- обе женщины, одна четвертая --- оба мужчины и одна вторая --- один мужчина и одна женщина. Среди рождений в США примерно 1 из 90 --- рождение близнецов. Определим следующие события:
\begin{gather*}
    A = \{\text{В результате родов в США рождаются женщины близнецы}\} \\
    B = \{\text{В результате родов в США рождаются однояйцевые близнецы}\} \\
    C = \{\text{В результате родов в США рождаются близнецы}\}
\end{gather*}
\begin{enumerate}
    \item Изложите словами событие $A \cap B \cap C$
    \item Найдите $P(A \cap B \cap C)$
\end{enumerate}
\end{problem}
\begin{solution}
    \begin{enumerate}
        \item В результате родов в США рождаются однояйцевые женщины близнецы.
        \item По определению условной вероятности:
            \begin{gather*}
            P(A \cap B \cap C) = P(A \cap B ~|~ C)P(C) = P(A  ~|~ B,C) P(B ~|~ C)P(C) = \\ 
            = \frac{1}{2} \cdot \frac{1}{3} \cdot \frac{1}{90} = \frac{1}{540}
            \end{gather*}
    \end{enumerate}
\end{solution}

\begin{problem}
    Пусть два игрока $A$ и $B$ поочередно и независимо подбрасывают монетку. Первый игрок, получивший орла --- выигрывает. Пусть $A$ кидает монетку первым.
    \begin{enumerate}
        \item Пусть монетка честная. Какова вероятность того, что игрок $A$ выиграет?
        \item Пусть $P(\text{Выпал орел}) = p$. Какова вероятность того, что игрок $A$ выиграет?
        \item Покажите, что для любого $p ~(0 < p < 1)$ выполняется: $P(A\text{ побеждает}) > 1/2$
    \end{enumerate}
\end{problem}
\begin{solution}
    \begin{enumerate}
        \item Представим игру в виде дерева, где в вершинах находится событие, а в ребрах вероятность перейти к событию: 
        \begin{tikzpicture}[
            frac/.style={sloped, midway, above, font=\small}
        ]
            \node (root) at (0,0) {};
            \node[] (L1) at (-2,-2) {О}; % Текст здесь
    
            \node (R1) at (1.5,-1) {Р};
            \node[] (L2) at (0.5,-3) {О}; % И здесь
    
            \node (R2) at (2.7,-1.8) {Р};
            \node[] (L3) at (2,-3.8) {О}; % И здесь
    
            \node (R3) at (3.7, -2.5) {Р};
            \node (L4_end) at (3.2, -4.5) {};
            \node (R4_end) at (4.2, -4.5) {};

            \draw[->] (root) -- (L1) node[frac, pos=0.4] {$\frac{1}{2}$};
            \draw[->] (root) -- (R1) node[frac, pos=0.4] {$\frac{1}{2}$};
    
            \draw[->] (R1) -- (L2) node[frac, pos=0.4] {$\frac{1}{4}$};
            \draw[->] (R1) -- (R2) node[frac, pos=0.6] {$\frac{1}{4}$};
    
            \draw[->] (R2) -- (L3) node[frac, pos=0.5] {$\frac{1}{8}$};
            \draw[->] (R2) -- (R3) node[frac, pos=0.5] {$\frac{1}{8}$};
    
            \draw[->] (R3) -- (L4_end) node[frac, pos=0.5] {$\frac{1}{24}$};
            \draw[->] (R3) -- (R4_end) node[frac, pos=0.5] {$\frac{1}{24}$};

            \node at (L4_end) [below=1mm] {$\dots$};
            \node at (R4_end) [below=1mm] {$\dots$};
    
        \end{tikzpicture} 

        Т.к. игрок $A$ играет первым, нам необходимо посчитать сумму ряда:
        \[\frac{1}{2} + \frac{1}{8} + \frac{1}{32} + \dots\]

        Это бесконечная убывающая геометрическая прогрессия, поэтому ее сумма равна:
        \[\frac{\frac{1}{2}}{1 - \frac{1}{4}} = \frac{2}{3}\]
        
        \item Аналогично, имеем ряд:
        \[p (1 + (1-p)^2 + (1-p)^4 + \dots)\]
        
        И получаем:
        \[p \cdot \frac{1}{1 - (1 - p)^2}\]
        
        \item Из второго пункта достаточно показать, что:
            \[\frac{p}{1 - (1-p)^2} > \frac{1}{2}\]

            В итоге имеем:
            \[\frac{p}{1 - (1-p)^2} = \frac{1}{2 - p} > \frac{1}{2} ~~~\forall p ~(0 < p < 1)\]

    \end{enumerate}
\end{solution}

\begin{problem}
    Семья Смитов имеет двух детей. Как минимум один из них --- мальчик. Какая вероятность того, что двое детей --- мальчики?
\end{problem}
\begin{solution}
    
\end{solution}

\begin{problem}
    Честный кубик бросается до тех пор, пока не выпадет число $6$. Какова вероятность того, что для этого его необходимо бросить более пяти раз?
\end{problem}
\begin{solution}
    
\end{solution}

\section{Случайные величины, функции распределения и плотности}

Здесь основными концепциями являются: случайная величина, функция распределения, функция и плотность вероятности.

\section{Аналитические и продвинутые задачи}

Здесь основными концепциями являются: геометрическая вероятность, математические тождества, анализ свойств функций, формула включений-исключений, аппроксимации.

\begin{problem}
    Докажите следующие тождества:
    \begin{enumerate}
        \item $\sum_{k=0}^n (-1)^k \binom{n}{k} = 0$
        \item $\sum_{k=1}^n k \binom{n}{k} = n 2^{n-1}$
        \item $\sum_{k=1}^n (-1)^{k+1} k \binom{n}{k} = 0$ 
    \end{enumerate}
\end{problem}
\begin{solution}
    
\end{solution}

\begin{problem}
    Докажите, что:
    \[\lim_{n \to \infty}\frac{n!}{n^{n + (1/2)}e^{-n}} = \text{constant}\]
\end{problem}
\begin{solution}
    
\end{solution}

\end{document}
