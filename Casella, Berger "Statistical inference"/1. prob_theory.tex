% !TeX root = ./document.tex
\documentclass[document]{subfiles}
\begin{document}
\chapter{Теория вероятностей}
\section{Теория множеств}

\begin{definition}
    Множество $S$ всех возможных исходов эксперимента называется пространством элементарных исходов для этого эксперимента.
\end{definition}

Будем разделять счётные и несчётные пространства элементарных исходов.

\begin{definition}
    Событием называется любое подмножество $S$ для конкретного эксперимента.
\end{definition}

\begin{remark}
    Будем обозначать дополнение события как $A^c$. Остальные правила работают для событий так же как и для множеств.
\end{remark}

\begin{definition}
    События $A$ и $B$ называются несовместными, если $A \cap B = \varnothing$. События $A_1, \dots, A_n$ называются попарно несовместными, если $A_i \cap A_j = \varnothing, ~ i \ne j$
\end{definition}

\begin{definition}
    Пусть $A_1, \dots, A_n$ попарно несовместные события и $\bigcup_{i=1}^\infty A_i = S$. Тогда эти события образуют разбиение $S$.
\end{definition}

\section{Базовая теория вероятностей}

\begin{definition}
    Совокупность подмножеств $\mathcal{B}$ множества $S$ называется сигма-алгеброй (или борелевским полем) если удовлетворяет следующим свойствам:
    \begin{enumerate}
        \item $\varnothing \in \mathcal{B}$
        \item $A \in \mathcal{B} \implies A^c \in \mathcal{B}$
        \item $A_1, A_2, \dots \in \mathcal{B} \implies \bigcup_{n=1}^\infty A_n \in \mathcal{B}$
    \end{enumerate}
\end{definition}

На самом деле этих свойств достаточно для того, чтобы сигма-алгебра была замкнута относительно счётного пересечения.

\begin{definition}
    Пусть $S$ --- пространство элементарных исходов и $\mathcal{B}$ --- его сигма-алгебра. Тогда вероятностью (вероятностной мерой) $P$ этого пространства будем называть функцию, удовлетворяющую следующим свойствам:
    \begin{enumerate}
        \item $\forall A \in \mathcal{B}P(A) \ge 0$
        \item $P(S) = 1$
        \item Пусть $A_1, A_2, \dots \in \mathcal{B}$ попарно несовместны, тогда $$P(\bigcup_{n=1}^\infty A_n) = \sum_{n=1}^\infty P(A_n)$$
    \end{enumerate}
\end{definition}

\begin{theorem}
    Пусть $S = \{s_1, \dots, s_n\}$ --- конечное пространство элементарных исходов. Пусть $\mathcal{B}$ --- любая сигма-алгебра на этом пространстве. Положим неотрицательные числа $p_1, \dots, p_n$ такие, что $p_1 + \dots + p_n = 1$. Для каждого $A \in \mathcal{B}$ положим функцию:
    \[P(A) = \displaystyle\sum_{i: s_i \in A} p_i\]
    Тогда функция $P$ является вероятностью. Это утверждение верно и для счётного множества $S = \{s_1, s_2, \dots\}$
\end{theorem}
\begin{proof}
    \begin{enumerate}
        \item Для любых $A \in \mathcal{B}$ имеем:
            \[\forall i: p_i \ge 0 \implies P(A) = \sum_{i: s_i \in A} p_i \ge 0\]
        \item Имеем:
            \[P(S) = \sum_{i: s_i \in S} p_i = 1\]
        \item Пусть $A_1, \dots, A_k$ --- попарно несовместные события, тогда из этого и из определения $P$:
            \[P \left( \bigcup_{i=1}^k A_i \right) = \sum_{j: s_j \in \bigcup_{i=1}^k A_i} p_j = \sum_{i=1}^k \sum_{j: s_j \in A_i} p_j= \sum_{i=1}^k P(A_i)\]
    \end{enumerate}
\end{proof}

\begin{theorem}
    Пусть $P$ --- вероятность и $A \in \mathcal{B}$ --- событие. Тогда:
    \begin{enumerate}
        \item $P(\varnothing) = 0$
        \item $P(A) \le 1$
        \item $P(A^c) = 1 - P(A)$
    \end{enumerate}
\end{theorem}
\begin{proof}
    Сначала докажем третье утверждение. По определению дополнения имеем $S = A \cup A^c$. Тогда по 2 и 3 аксиоме вероятностей:
    \[P(A \cup A^c) = P(A) + P(A^c) = P(S) = 1\]

    Т.к. $P(A^c)\ge 0 $ то сразу имеем верность второго пункта.

    Третий пункт доказывается аналогично:
    \[P(S) = P(S \cup \varnothing) = P(S) + P(\varnothing) \implies P(\varnothing) = 0\]
\end{proof}

\begin{theorem}
    Пусть $P$ --- вероятность и $A,B \in \mathcal{B}$ --- события. Тогда:
    \begin{enumerate}
        \item $P(B \cap A^c) = P(B) - P(A \cap B)$
        \item $P(A \cup B) = P(A) + P(B) - P(A \cap B)$
        \item Пусть $A \subset B$ тогда $P(A) \le P(B)$
    \end{enumerate}
\end{theorem}
\begin{proof}
    \begin{enumerate}
        \item Имеем:
        \[B = (B\cap A) \cup (B \cap A^c)\]
        Тогда из того что $B \cap A$ и $B \cap A^c$ попарно несовместны имеем:
            \[P(B) = P(B \cap A) + P(B \cap A^c)\]
        \item Имеем:
            \[A \cup B = A \cup (B \cap A^c)\]
            Тогда из того, что $A$ и $B \cap A^c$ несовместны и по первому пункту имеем:
            \[P(A \cup B) = P(A) + P(B \cap A^c) = P(A) + P(B) - P(A \cap B)\]
        \item Если $A \subset B$ тогда $A \cap B = A$ и по первому пункту имеем:
            \[0 \le P(B \cap A^c) = P(B) - A\]
    \end{enumerate}
\end{proof}
\begin{definition}
    Из первого пункта теоремы и того, что $P(B \cap A^c) \le 1$ имеем неравенство Бонферрони:
    \[P(A \cap B) \ge P(A) + P(B) - 1\]
\end{definition}

\begin{theorem}
    Пусть $P$ --- вероятность, тогда:
    \begin{enumerate}
        \item $P(A) = \sum_{n=1}^\infty P(A \cap C_n)$ для любого разбиения $C_1, C_2, \dots$
        \item $P(\bigcup_{n=1}^\infty A_n) \le \sum_{i=1}^\infty P(A_n)$ для любых $A_1, A_2, \dots$ (Неравенство Буля)
    \end{enumerate}
\end{theorem}
\begin{proof}
    \begin{enumerate}
        \item Т.к. $C_1, C_2, \dots$ --- разбиение, то $C_i \cap C_j = \varnothing, ~ i \ne j$ и $S = \bigcup_{n=1}^\infty C_n$. Тогда:
            \[A = A \cap S = A \cap \left(\bigcup_{n=1}^\infty C_n\right) = \bigcup_{n=1}^\infty (A \cap C_n)\]
            Т.к. $A \cap C_i$ --- попарно несовместны то по свойствам вероятности:
            \[P\left(\bigcup_{n=1}^\infty (A \cap C_n)\right) = \sum_{n=1}^\infty P(A \cap C_n)\]
        \item Построим несовместные события $A_1^*, A_2^*, \dots$ следующим образом:
            \[A_1^* = A_1, \quad A_i^* = A_i \setminus \left( \bigcup_{j-1}^{i-1}A_j\right), i=2,3,\dots\]
            Тогда:
            \[P\left(\bigcup_{i=1}^\infty A_i\right) = P\left(\bigcup_{i=1}^\infty A_i^*\right) = \sum_{i=1}^\infty P(A_i^*)\]
            Так же по построению $A_i^* \subset A_i$ а значит:
            \[\sum_{i=1}^\infty P(A_i^*) \le \sum_{i=1}^\infty P(A_i)\]
    \end{enumerate}
\end{proof}

Можем применить предыдущую теорему к $A^c$ получая:
\[P\left(\bigcup_{i=1}^n A_i^c \right) \le \sum_{i=1}^n P(A_i^c)\]

Используя то, что $\bigcup_{i=1}^n A^c_i = (\bigcap_{i=1}^n A_i)^c$ и $P(A_i^c) = 1 - P(A_i)$ имеем:
\[1 - P \left(\bigcap_{i=1}^n A_i\right) \le n - \sum_{i=1}^n P(A_i)\]

или, если переставить слагаемые:
\[P\left( \bigcap_{i=1}^n A_i\right) \ge \sum_{i=1}^n P(A_i) - (n - 1)\]

\section{Условная вероятность и независимость событий}

До этого мы полагали, что события не зависят друг от друга. Поэтому введём понятие зависимых событий.

\begin{definition}
    Пусть $A, B$ --- события в пространстве элементарных событий $S$ и $P(B) > 0$, тогда вероятностью события $A$ при условии $B$ является:
    \[P(A ~|~ B) = \frac{P(A \cap B)}{P(B)}\]
\end{definition}

Формулу можно переписать следующим образом:
\[P(A \cap B) = P(A~|~B) P(B)\]

Или, пользуясь симметричностью формулы:
\[P(A \cap B) = P(B ~|~ A) P(A)\]

Таким образом,, можно получить формулу Байеса:
\[P(A~|~B) = P(B ~|~ A)\frac{P(A)}{P(B)}\]

\begin{theorem}
    Пусть $A_1, A_2, \dots$ --- разбиение пространства элементарных исходов и $B$ --- любое событие. Тогда для каждого $i = 1,2, \dots$ верна формула:
    \[P(A_i ~|~ B) = \frac{P(B~|~ A_i) P(A_i)}{\sum_{j=1}^\infty P(B ~|~ A_j) P(A_j)}\]
\end{theorem}

\begin{definition}
    Пусть $A,B$ --- события. Мы будем называть их независимыми, если:
    \[P(A \cap B) = P(A)P(B)\]
\end{definition}
\begin{remark}(Откуда это определение взялось?)
    Пусть $A$ не зависит от $B$, тогда логично положить, что $P(A ~|~B) = P(A)$, тогда имеем:
    \[P(A \cap B) = P(A~|~B)P(B) = P(A)P(B)\]
    Тогда мы получаем сразу, что $P(B ~|~ A) = P(B)$.
\end{remark}

\begin{theorem}
    Пусть $A$ и $B$ --- независимые события. Тогда независимыми являются так же:
    \begin{enumerate}
        \item $A$ и $B^c$
        \item $A^c$ и $B$
        \item $A^c$ и $B^c$
    \end{enumerate}
\end{theorem}
\begin{proof}
    Докажем только первый пункт, остальные будут в задачах. Для этого нам надо показать, что $P(A \cap B^c) = P(A)P(B^c)$
    По теореме $1.3$ мы имеем:
    \begin{gather*}
        P(A \cap B^c) = P(A) - P(A \cap B) = P(A) - P(A)P(B) = \\ = P(A)(1 - P(B)) = P(A)P(B^c)
    \end{gather*}
\end{proof}

\begin{definition}
    Множество событий $A_1, \dots, A_n$ называются независимыми в совокупности, если для любого подмножества событий $A_{i_1}, \dots, A_{i_k}$ выполняется:
    \[P \left(\bigcap_{j=1}^k A_{i_j}\right) = \prod_{j=1}^k P(A_{i_j})\]
\end{definition}

\section{Случайные величины}

\begin{definition}
    Случайной величиной будем называть функцию $X: S \to \mathbb{R}$
\end{definition}

Случайная величина задаёт некоторое пространство элементарных исходов, на котором так же нужно определить вероятность.

Положим пространство элементарных исходов:
\[S = \{s_1, \dots, s_n\}\]
с вероятностью $P$.

Определим случайную величину $X$ со значениями $\mathcal{X} = \{x_1, \dots, x_m\}$ тогда можем определить вероятность:

\begin{definition}
    Индуцированной функцией вероятности $P_X$ на пространстве $\{x_1, \dots, x_m\}$ будем называть функцию:
    \[P_X(X = x_i) = P(\{s_j \in S: X(s_j) = x_i\})\]
\end{definition}
На самом деле это определение неявно подразумевает, что случайная величина должна быть измеримой функцией, то есть прообраз любого измеримого на $\mathbb{R}$ подмножества должно лежать в сигма-алгебре пространства $S$.

\section{Функции распределений}

С каждой случайной величиной будем ассоциировать следующий объект:
\begin{definition}
    Функцией распределения случайной величины $X$, будем называть функцию:
    \[F_X(x) = P_X(X \le x), \quad \forall x \in \mathbb{R}\]
\end{definition}

\begin{theorem}
    Функция $F(x)$ является функцией распределения тогда и только тогда, когда выполняются следующие условия:
    \begin{enumerate}
        \item $\lim_{x \to -\infty} F(x) = 0$ и $\lim_{x \to +\infty}F(x) = 1$
        \item $F(x)$ неубывающая функция по $x$
        \item $F(x)$ непрерывна справа
    \end{enumerate}
\end{theorem}
\begin{proof}
    Не будет
\end{proof}

\begin{definition}
    Случайную величину $X$ будем называть непрерывной, если непрерывна ее функция распределения, иначе --- дискретной.
\end{definition}

Если взять самую маленькую сигма-алгебру $\mathcal{B}^1$ содержащую все открытые интервалы $(a,b)$, то функция вероятности однозначно определяет случайную величину. В общем случае это, конечно же, не так.

\begin{definition}
    Случайные величины $X$ и $Y$ называются одинаково распределёнными, если для любого $A \in \mathcal{B}^1: P(X \in A) = P(Y \in A)$
\end{definition}

\begin{theorem}
    Следующие два утверждения эквивалентны:
    \begin{enumerate}
        \item Случайные величины $X$ и $Y$ одинаково распределены
        \item $F_X(x) = F_Y(x) \quad \forall x$
    \end{enumerate}
\end{theorem}
\begin{proof}
    Докажем $1 \implies 2$:

    Т.к. $X$ и $Y$ одинаково распределены, то взяв $A = [-\infty, x) \quad \forall x$ имеем:
    \[F_X(x) = P(X \in [-\infty, x)) = P(Y \in [-\infty, x)) = F_Y(x)\]
    
    Докажем $2 \implies 1$

    В книге не представлено.
\end{proof}

\section{Функция плотности и функция вероятностей}

\begin{definition}
    Функцией вероятности случайной величины $X$ называется функция:
    \[f_X(x) = P(X = x) \quad \forall x\]
\end{definition}

\begin{definition}
    Плотностью вероятности непрерывной случайной величины называется такая функция $f_X$, удовлетворяющая равенству:
    \[F_X(x) = \int_{-\infty}^x f_X(t)dt \quad \forall x\]
\end{definition}

\begin{theorem}
    Функция $f_X(x)$ является плотностью вероятности (или функцией вероятности) некоторой случайной величины тогда и только тогда, когда выполняются условия:
    \begin{enumerate}
        \item $f_X(x) \ge 0 \quad \forall x$
        \item $\sum_x f_X(x) =1 ~\text{(функция вероятности)}$ или 

            $\int_{-\infty}^{+\infty}f_X(x)dx = 1 ~ \text{(плотность вероятности)}$
    \end{enumerate}
\end{theorem}
\begin{proof}
    Изич
\end{proof}

\end{document}
